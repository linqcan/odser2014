\begin{abstract}
The increasing complexity of software systems has lead to increased demands on the tools and methods used when developing software systems. To determine if a tool or method is more efficient or accurate than others empirical studies are used. The data used in empirical studies might be affected by outliers i.e. data points that deviates significantly from the rest of the data set. Hence, the statistical analysis might be distorted by these outliers as well. 


This study investigates if outliers are present within Empirical Software Engineering (ESE) studies using unsupervised methods for detection. It also tries to assess if the statistical analysis performed in ESE studies are affected by outliers by removing them and performing a re-analysis. The subjects used in this study comes from narrow literature review of recently published papers within Software Engineering (SE). While collecting the samples needed for this study the current state of practise regarding data availability and analysis reproducibility is investigated. 


This study's results shows that outliers can be found in ESE studies and it also identifies issues regarding data availability within the same field. Finally, this study presents guidelines for how to improve the way outlier detection is presented within ESE studies as well as guidelines for publishing data.


\end{abstract}
